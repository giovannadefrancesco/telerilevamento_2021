\documentclass[10pt]{beamer}
%\documentclass[aspectratio=169, 16pt]{beamer} 
\usepackage[utf8]{inputenc}
\usetheme[progressbar=frametitle]{Goettingen}
\usecolortheme{default}
\usepackage{listings}
\setbeamertemplate{blocks} [rounded] [shadow=true]


\title{Abbassamento del livello di acqua nel
              \\Lago Powell dal 2017 al 2021}
  \subtitle {\footnotesize corso \\ Telerilevamento Geo-Ecologico.}
  \author{Giovanna De Francesco}
  \institute {Alma Mater Studiorum University of Bologna\\ 
  \bigskip
  \includegraphics[width=0.4\textwidth]{figure/logo_unibo.png}}
  \date{26 luglio 2022}

\begin{document}
   \maketitle

 \AtBeginSection[]
{   
\begin{frame}
\frametitle{Indice}
\tableofcontents[currentsection,currentsubsection,currentsubsubsection]
\end{frame}
}

\section {Introduzione}
\begin{frame} 
    \begin{itemize}
      \item \textbf{Area di studio}
    \end{itemize}
            {\scriptsize Il lago Powell è il secondo bacino artificiale più grande degli Stati Uniti, si sviluppa lungo il fiume Colorado tramite la costruzione della diga di Glen Canyon.
            \\Si trova a confine tra il Sud/Est dello Utah e il Nord/Est dell’Arizona ed è un'importante riserva idrica.\par}
    \begin{figure}
      \centering
      \includegraphics[width=\textwidth, height=0.6\textheight,keepaspectratio]{figure/Area di studio.jpg}
    \end{figure}
\end{frame}


\begin{frame} 
    \begin{itemize}
      \item \textbf{Confronto lago Powell 2017 vs 2021}
    \end{itemize}
            {\scriptsize Negli ultimi anni, il livello del lago Powell è sceso drasticamente, arrivando a registrare il dato più basso da quando è stato riempito nel 1980. 
            \\Questo drastico calo del livello dell’acqua è  stato documentato da immagini catturate dal satellite Landsat8 riferite al 2017 e al 2021 .\par}
            \bigskip 
    \begin{figure}
      \centering
      \includegraphics[width=\textwidth, height=0.7\textheight,keepaspectratio]{figure/Confronto lago powel 2017 e 2021.png}
    \end{figure}
\end{frame}


\begin{frame} 
    \begin{itemize}
      \item \textbf{Differenze evidenziate}
    \end{itemize}
    \begin{figure}
      \centering
      \includegraphics[width=\textwidth, height=0.7\textheight,keepaspectratio]{figure/Differenza 2021-2017.png}
    \end{figure}
    \begin{center}
      {\large Vengono evidenziate in \textcolor{red}{rosso} le differenze maggiori.
      \\La restituzione di questa immagine indica quali sono state le zone in cui c'è stato un ritiro maggiore dell'acqua.\par}
    \end{center}
\end{frame}


\section{Classificazione}
\begin{frame} 
    \begin{itemize}
      \item \textbf{UnsuperClasse}
            \\{\scriptsize La classificazione non supervisionata permette di individuare zone che hanno valori simili di riflettanza e li ragruppa in classi. Codice per l'immagine del 2017\par}
    \begin{mybox}
      \tiny\lstinputlisting[language=R]{Unsuperclasse.txt}
    \end{mybox}
    \end{itemize}
\end{frame}


\begin{frame} {}
     \centering
     \\{Le immagini mostrano le classi a cui i pixel sono stati assegnati. 
     \\L'assegnazione avviene in modalità random.\par}
   \begin{figure}
      \centering
      \includegraphics[width=\textwidth, height=0.6\textheight,keepaspectratio]{figure/Unsuperclass.jpg}
    \end{figure}
    \begin{itemize}
      \centering
         \item 2017: l'area bagnata corrisponde alla calsse 2.
         \item 2021: l'area bagnata corrisponde alla calsse 1.
    \end{itemize}
\end{frame}
 

\begin{frame} {}
    \begin{itemize}
       \item \textbf{Confronto delle classi tramite grafici a barre.}
    \end{itemize}
    \begin{figure}
      \centering
      \bigskip 
      \includegraphics[width=\textwidth, height=0.6\textheight,keepaspectratio]{figure/unsuperclass_barre.png}
    \end{figure}
\end{frame}


\begin{frame} {}
    \begin{mybox}
      \tiny\lstinputlisting[language=R]{percentuale acqua.txt}
   \end{mybox}
      \bigskip 
      \centering
      \\{Dato l'obiettivo del progetto, si è scelto di operare solo sul numero di pixel delle classi riferite all'area bagnata per le due immagini.\par} 
      \bigskip 
      {Si evince che la riduzione in percentuale dell'area baganta rispetto al 2017 è di circa \textcolor{red}{\large 31\%}.\par}
\end{frame}


\section {Indici di Analisi}
\subsection {NDWI}
\begin{frame} 
    \begin{itemize}
      \item \textbf{NDWI: Normalized Difference Water Index}
      \\{\scriptsize Si tratta di un indice che viene utilizzato per monitorare i cambiamenti relativi al contenuto idrico nei corpi idrici. Poiché i corpi idrici assorbono fortemente la luce nello spettro elettromagnetico da visibile a infrarosso, NDWI utilizza le bande del verde e del vicino infrarosso per evidenziare i corpi idrici.\par} 
    \begin{equation}
       NDWI={\frac{(GREEN - NIR)}{(GREEN + NIR)}}
    \end{equation}
    \begin{mybox}
       \tiny\lstinputlisting[language=R]{RcodeNDVI.txt}
    \end{mybox}
    \end{itemize}
\end{frame}


\begin{frame} {}
    \begin{itemize}
      \item \textbf{Confronto NDWI tra il 2017 e il 2021.}
    \end{itemize}
    \begin{figure}
      \centering
      \bigskip 
      \includegraphics[width=\textwidth, height=0.7\textheight,keepaspectratio]{figure/confrontoNDVI.PNG}
    \end{figure}
\end{frame}


\begin{frame} {}
    \begin{itemize}
      \item \textbf{Differenza NDWI 2021-2017}
    \end{itemize}
    \begin{figure}
      \bigskip 
      \includegraphics[width=\textwidth, height=0.6\textheight,keepaspectratio]{figure/Differenza NDWI.PNG}
      {\\ Le zone a sfumature di \textcolor{blue}{ blu} indicano la maggior differenza dell'indice NDWI negli anni in esame, ovvero le zone dove è avvenuto maggior ritiro del livello dell'acqua.\par}
    \end{figure}
\end{frame}


\subsection {SAVI}
\begin{frame}
    \begin{itemize}
      \item \textbf{SAVI: Soil-Adjusted Vegetation Index}
      \\{\scriptsize E' un indice adoperato per correggere l'NDVI (indice che descrive il livello di vigoria della vegetazione) tenendo conto dell'influenza della luminosità del suolo nelle aree con scarsa copertura vegetativa.
      \\ Viene utilizzato per l'analisi delle regioni aride con vegetazione rada (meno del 15\% dell'area totale) e superfici del suolo esposte.
      \\La correzione viene valutata tramite il parametro L, fattore di correzione della luminosità del suolo.\par} 

    \begin{equation}
       SAVI={\frac{((NIR - RED)}{(NIR + RED + L)}}*{(1+L)}
    \end{equation}

      \\{\scriptsize # NIR e RED si riferiscono alle bande associate a tali lunghezze d'onda. 
      \\Il valore L varia a seconda della quantità di copertura di vegetazione verde:
      \\-nelle aree senza copertura di vegetazione verde, L=1;
      \\-nelle aree a moderata copertura di vegetazione verde, L=0,5;
      \\-nelle aree a elevata copertura di vegetazione verde, L=0.
      \bigskip
      \\In questo caso specifico, dato che nelle vicinanze del lago Powell non è presente vegetazione, uso L=1.\par}
    \end{itemize}
\end{frame}


\begin{frame} {}
    \begin{itemize}
      \item \textbf{Confronto SAVI tra il 2017 e il 2021.}
    \end{itemize}
    \begin{figure}
      \centering
      \bigskip 
      \includegraphics[width=\textwidth, height=0.7\textheight,keepaspectratio]{figure/Confronto SAVI.PNG}
    \end{figure}
\end{frame}


\begin{frame} {}
    \begin{itemize}
      \item \textbf{Differenza SAVI 2021-2017}
    \end{itemize}
    \begin{figure}
      \bigskip 
      \includegraphics[width=\textwidth, height=0.6\textheight,keepaspectratio]{figure/Differenza SAVI.PNG}
      {\\Le zone \textcolor{red}{rosse} rappresentano la maggior differenza dell'indice SAVI negli anni esaminati. Ciò è sinonimo del fatto che, a causa della diminuzione del livello dell'acqua del lago Powell, è aumentata l'esposizione del suolo.\par}
    \end{figure}
\end{frame}


\section {PCA}
\begin{frame}
    \begin{itemize}
      \item \textbf{PCA: Principal Component Analysis}
      {\\L'Analisi delle componenti principali è una tecnica che semplifica i dati derivanti da più variabili in un numero minore di variabili, limitando la perdita delle informazioni.
      \bigskip 
      \\ Più precisamente, consiste nel prendere i dati originali e si fa passare un asse lungo la variabilità maggiore e l'altro asse lungo la variabilità minore.\par} 
    \end{itemize}
\end{frame}

\begin{frame}
    \begin{itemize}
      \item \textbf{Plot delle tre bande dell'immagine del 2017}
    \end{itemize}
    \begin{figure}
      \bigskip 
      \includegraphics[width=\textwidth, height=0.7\textheight,keepaspectratio]{figure/pca_2017.PNG}
    \end{figure}
\end{frame}


\begin{frame} {}
    \begin{itemize}
      \item \textbf{Plot delle tre bande dell'immagine del 2021}
    \end{itemize}
    \begin{figure}
      \bigskip 
      \includegraphics[width=\textwidth, height=0.7\textheight,keepaspectratio]{figure/pca_2021.PNG}
    \end{figure}
\end{frame}


\begin{frame} {}
   \begin{itemize}
     \item \textbf{Correlazione fra le bande del 2017 e del 2021}
   \end{itemize}
   \begin{figure}
      \bigskip 
      \includegraphics[width=\textwidth, height=0.7\textheight,keepaspectratio]{figure/correlazione.PNG}
   \end{figure}
\end{frame}    


\begin{frame} {}
   \begin{itemize}
     \item \textbf{Codice calcolo PCA per analisi multivariata}
   \end{itemize}
   \begin{mybox}
     \tiny\lstinputlisting[language=R]{analisiPCA.txt}
    \end{mybox}
\end{frame}    


\begin{frame} {}
    \begin{itemize}
      \item \textbf{Output PCA}
    \end{itemize}
    \begin{figure}
      \bigskip 
      \centering
      \includegraphics[width=\textwidth, height=0.5\textheight,keepaspectratio]{figure/pca finale.PNG}
    \end{figure}
    \centering
    {\\Immagini risultanti dall'analisi delle componenti principali. \par}
    
     \bigskip
    {\\Grazie all'analisi PCA si sono create delle nuove componenti che diminuiscono la correlazione iniziale
    fra tutte le bande e con un numero minore di componenti, è possibile spiegare l'immagine originale, togliendo la correlazione.\par}
\end{frame}    


\section {}
\subsection {Variabilità locale}
\begin{frame} 
    \begin{itemize}
      \item \textbf{R code Variabilità sulla PC1}
    \end{itemize}
    \begin{mybox}
      \tiny\lstinputlisting[language=R]{variabilita.txt}
    \end{mybox}
\end{frame}


\begin{frame} {}
    \begin{itemize}
      \item \textbf{Deviazione standard derivante dalla PC1 del 2017 e del 2021}
    \begin{figure}
      \bigskip 
      \includegraphics[width=\textwidth, height=0.39\textheight,keepaspectratio]{figure/sd_finale.PNG}
    \end{figure}
    \end{itemize}
\end{frame}


\section {Spectral Signature}
\begin{frame}
    \begin{itemize}
      \item \textbf{Spectral Signature}
      {\\Le \textbf{FIRME SPETTRALI} servono per vedere se quello che si vede in termini di colori e quindi, di differenziazione nei tempi, corrisponde effettivamente a una variazione della riflettanza.
      \bigskip 
      \\ Vengono viste come una sorta di "impronte digitali" delle varie componenti che formano un qualsiasi ecosistema.
\par} 
    \end{itemize}
\end{frame}


\begin{frame}
    \begin{itemize}
      \item \textbf{Analisi multitemporale}
    \end{itemize}
      \centering
      {Sono stati selezionati 5 punti nell'immagine del 2017 e 5 punti nell'immagine del 2021, aventi circa le stesse coordinate e sono stati utilizzati per ricavare le varie firme spettrali e fare l'analisi di multitemporalità.}
    \begin{figure}
      \bigskip 
      \includegraphics[width=\textwidth, height=0.35\textheight,keepaspectratio]{figure/sceltapunti.PNG}
    \end{figure}
     {\scriptsize Sono evidenziati in \textcolor{magenta}{magenta} i punti su cui si è calcolata la variabilità.\par}
\end{frame}


\begin{frame} {}
    \begin{itemize}
      \item \textbf{R code Spectral Signature}
    \end{itemize}
    \begin{mybox}
      \tiny\lstinputlisting[language=R]{spectral signature.txt}
    \end{mybox}
\end{frame}


\begin{frame} {}
    \begin{itemize}
      \item \textbf{Creazione DataFrame}
    \end{itemize}
    \begin{figure}
       \bigskip 
       \includegraphics[width=\textwidth, height=0.4\textheight,keepaspectratio]{figure/dataframe.PNG}
    \end{figure}
\end{frame}


\begin{frame} {}
    \begin{itemize}
      \item \textbf{Risultati analisi multitemporale}
    \end{itemize}
    \begin{figure}
      \bigskip 
      \includegraphics[width=\textwidth, height=0.7\textheight,keepaspectratio]{figure/Risultati_multitemporalePNG.PNG}
    \end{figure}
\end{frame}

\section {Conclusioni}
\begin{frame} 
    \begin{itemize}
      \item Grazie alla funzione \emph{UnsuperClass} si è stimato che il prosciugamento del lago Powell dal 2017 al 2021 sia del 31\%.
      \item Gli indici {\textbf{NDWI e SAVI}}  hanno dimostrato che c'è stata:
      \\-una diminuzione del livello dell'acqua;
      \\-un aumento dell'esposizione del suolo dovuto al prosciugamento.
      \item Elaborazione di mappe di variabilità locale utilizzando la PC1 per il 2017 e il 2021.
      \item Le \emph{Spectral Signature} confermano l'esposizione del suolo a causa dell' assenza di acqua.
    \end{itemize}
\end{frame}

\begin{frame} {}
  \centering
  \textbf{\huge Grazie per l'attenzione!\par}
\end{frame}
\end{document}
